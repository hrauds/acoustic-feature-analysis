The aim of this bachelor's thesis was to create a desktop application for analyzing acoustic features of GeMAPS. The application allows extracting GeMAPS features from audio files, visualizing them on different chart types (timeline, histogram, box, radar and vocal map charts) and similarity analysis to find the most similar recordings. Audio File Processing (OpenSMILE) uses data from TextGrids to allow specific phoneme and word analysis. The extracted feature values are stored in a MongoDB document-based database. The user interface is created in the PyQt Python framework and the Plotly library is used to generate interactive graphs. Both clustering (KMeans) and cosine similarity calculation are used as analysis methods.

The thesis is written in Estonian and is 34 pages long, including 6 chapters, 5 figures and 10 tables.