Käesoleva bakalaureusetöö eesmärk oli luua töölauarakendus, mis lihtsustab GeMAPS (eGeMAPS) akustiliste tunnuste kasutamist foneetilistes uuringutes. Rakendus võimaldab helifailidest GeMAPS-tunnuste eraldamist, nende visualiseerimist erinevatel diagrammitüüpidel (ajatelje-, histogrammi-, karp-, radari- ja vokaalikaardi diagrammid) ning sarnasuse analüüsi, et leida kõige sarnasemad salvestised. Helifailide töötlemisel (OpenSMILE) kasutatakse TextGridide andmeid, et võimaldada foneemide ja sõnade tasemel analüüsi. Tunnuste ajaraamilised väärtused talletatakse MongoDB dokumendipõhises andmebaasis. Kasutajaliides on loodud PyQt raamistikus ning interaktiivsete graafikute genereerimiseks rakendatakse Plotly teeki. Analüüsimeetoditena rakendatakse nii klasterdamist (KMeans) ja koosinussarnasuse arvutamist.

Lõputöö on kirjutatud eesti keeles ning sisaldab teksti [lehekülgede arv] leheküljel, [peatükkide arv] peatükki, [jooniste arv] joonist, [tabelite arv] tabelit.