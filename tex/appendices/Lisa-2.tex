Tere! Olen Hanna Raudsepp, Tallinna Tehnikaülikooli informaatika eriala bakalaureuseõppe tudeng. Minu lõputööks on "Hääle akustiliste tunnuste visualiseerimise rakendus" ning selle eesmärk on luua tööriist, mis võimaldab kõnesalvestuste akustiliste tunnuste analüüsi ja visuaaliseerimist.

Küsimustik on koostatud, et koguda tagasisidet rakenduse funktsionaalsuse ja kasutajamugavuse kohta.

Küsimused:

\begin{enumerate}
    \item Kas tegelete foneetika või kõne uurimisega?
    \item Kas rakenduse loodud graafikud olid arusaadavad ja kasulikud? (Rakenduse graafikud: ajagraafik, histogramm, karpdiagram, radar, vokaalikaart, sarnaste salvestuste klaster, sarnaste salvestuste tulpdiagrammid) 
    \item Kas märkasite rakenduse kasutamisel tehnilisi probleeme või tõrkeid? Kui jah, siis milliseid?
    \item Hinnake skaalal 1–5, kui kasutajasõbralik ja mugav oli rakenduse kasutamine teie arvates? 1 = „väga ebamugav“, 5 = „väga mugav“. Selgitage oma hinnangut.
    \item Millised rakenduse funktsioonid tundusid teie jaoks kõige kasulikumad?
    \item Milliseid täiustusi või lisafunktsioone soovitaksite rakendusele lisada?
    \item Muud kommentaarid
\end{enumerate}
