Järgnevalt antakse lühike ülevaade sarnase funktsionaalsusega olemasolevatest kõneanalüüsi tööriistadest. 

\textbf{Kõneveebi Audiofailide akustiliste omaduste võrdlus} \cite{kõneveeb} on veebirakendus, mis kasutab OpenSMILE eGeMAPS tunnuste komplekti kõnesalvestiste akustiliseks analüüsiks.

Rakenduse võimalused:
\begin{itemize}
    \item .wav-formaadis pakendatud audiofailide üleslaadimine, maksimaalne lubatud failide suurus 900MB
    \item Võimalik määrata klassifikaatoreid CSV-vormingus.
    \item Visualiseeritakse valitud akustiliste tunnuste väärtuste jaotus joondiagrammina
    \item Arvutatud väärtused ja üleslaetud failid kustutatakse peale brauseri sulgemist.
\end{itemize} 

\textbf{Praat} \cite{praat} on Amsterdami Ülikoolis arendatud vabavaraline kõneanalüüsi rakendus ning väga levinud foneetikauuringutes. Praat toetab Windows, macOS ja Linux operatsioonisüsteeme.
 
Rakenduse võimalused:
\begin{itemize}
    \item Akustiline analüüs ja erinevate tunnuste arvutamine (F0, formandid, intensiivsus, jitter, shimmer jpm)
    \item Visualiseerimisvõimalused: spektrogrammid, põhitooni ja intensiivsuse graafikud ajateljel, formantide trajektoorid
    \item Kõnesüntees: põhitooni ja formantide põhjal on võimalik luua modifitseeritud või sünteesitud kõne
    \item Skriptide loomise võimalus: Praatil on oma skriptikeel, mis teeb rakenduse paindlikuks, ning võimaldab automatiseerida korduvaid analüüse
\end{itemize}

\textbf{VoiceSauce} \cite{voicesauce} on MATLAB-i keskkonnas arendatud kõneanalüüsi rakendus. Rakendus on mõeldud teaduslikuks kõneanalüüsiks. Toetatud on Windows ja macOS operatsioonisüsteemid.

Rakenduse võimalused:
\begin{itemize}
    \item Erinevate akustiliste hääle tunnuste visualiseerimine ja analüüs.
    \item Mõned parameetrid sõltuvad tugevalt arvutatud F0 väärtustest, mistõttu võib analüüs olla ebatäpne mürarikaste helide korral
    \item Pikkade heliklippide analüüs võib tekitada ressursipuudust, kuna MATLABi keskkond võib intensiivsel analüüsil väga suurt osa mälu tarbida.
    \item Nõuab MATLABi kasutamist
\end{itemize} 

\textbf{Wasp (Windows Tool for Speech Analysis)} \cite{wasp} on tasuta kõneanalüüsi rakendus. Toetatud on operatsioonisüsteem Windows.

Rakenduse võimalused:
\begin{itemize}
    \item Helifailide salvestamine ja esitamine, salvestatud faile on võimalik analüüsida
    \item Võimaldab luua annotatsioone
    \item Akustiliste tunnuste visualiseerimine
    \item Lihtne ja kasutajasõbralik kasutajaliides ning saadaval on veebiversioon rakendusega tutvumiseks
\end{itemize} 

Kõik välja toodud rakendused (Kõneveebi veebirakendus, Praat, VoiceSauce ja Wasp) on võimelised kõnesalvestusi töötlema ning pakuvad akustiliste tunnuste arvutamist ja visualiseerimisvõimalusi. Neist kõige rohkem võimalusi ja funktsionaalsust on Praatil, kuid sellel ei ole spetsiaalset funktsionaalsust GeMAPS tunnuste eraldamiseks. Samuti on erinevate visualisatsioonide ja analüüside jaoks tavaliselt vajalik skriptide kirjutamine.

Selgelt erinev teistest on Kõneveebi lahendus, sest see on veebipõhine ning mõeldud ainult eGeMAPS-tunnuste analüüsiks. Selle funktsionaalsus on piiratud: rakenduses on küll võimalik akustilisi tunnuseid joondiagrammil visualiseerida, kuid puuduvad erinevad diagrammitüübid ning analüüsimeetodid sarnasuse otsinguks. Lisaks esineb failisuuruse piirang ja tulemusi ei salvestata.

Seega, ükski rakendus ei kata ühes keskkonnas kõiki käesolevas töös esile toodud analüüsiülesandeid: GeMAPS tunnuse arvutamine, erinevad visualiseerimis võimalused ning sarnasuse analüüsi.
