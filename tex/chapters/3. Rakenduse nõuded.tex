Järgnevalt kirjeldatakse rakenduse funktsionaalsete nõuete määratlemise protsessi ja esitatakse nende põhjal koostatud kasutusjuhud.

\section{Funktsionaalsed nõuded}
Rakenduse arenduse alguses lepiti juhendajaga kokku funktsionaalsed nõuded, et määratleda rakenduse fookus ja võimalused. 

Funktsionaalsed nõuded:
\begin{itemize}
    \item GeMAPS tunnuste eraldamine kõnefailidest. Rakendus peab suutma kõnefailidest eraldada GeMAPS akustilisi tunnuseid. Selleks tuleb töödelda salvestused ja nendega seotud TextGrid failid.
   
    \item Tunnuste visualiseerimine erinevatel diagrammitüüpidel, nagu ajatelje joondiagramm, histogramm, radardiagramm, karpdiagramm, vokaalikaart.
    
    \item Sarnaste kõnelejate leidmine kõnekorpusest.

    \item Võimalus eksportida analüüsi tulemusi: pildifaile ja andmeid.
\end{itemize}

\section{Funktsionaalsete nõuete kasutusjuhud}
Arenduse käigus koostati täpsemad nõuded, mida täiendati jooksvalt, kui juhendajaga koos rakenduse testides tekkis uusi ideid. Järgnevalt esitatakse kokkulepitud funktsionaalsed nõuded kasutusjuhtudena tabelites 1-7.

\begin{longtable}{|p{2.5cm}|p{11cm}|}
    \caption{Kasutusjuht GeMAPS tunnuste eraldamine kõnefailidest.}
    \label{tab:kasutusjuht1}\\ \hline
    \textbf{Nimi} &  \textbf{GeMAPS tunnuste eraldamine kõnefailidest}  \\
    \hline
    \endfirsthead
    \hline
    \textbf{Nimi} &  \textbf{GeMAPS tunnuste eraldamine kõnefailidest}  \\
    \hline
    \endhead
    \hline
    \endfoot
    \hline
    \endlastfoot
    Kirjeldus & Kasutaja saab importida kõnesalvestused ja nendega seotud TextGrid failid, et eraldada ja salvestada GeMAPS akustilisi tunnuseid.\\ \hline
    Eeltingimused & Kasutajal on kõnesalvestuste samanimelised .wav ja TextGrid failid.\\ \hline
    Põhiline sündmuste käik & 
    1. Kasutaja avab kõnefailide importimise akna.
    
    2. Süsteem kuvab kõnefailide ja TextGridide üleslaadimise vormi.
    
    3. Kasutaja valib kõnefailid ja vastavad TextGridid oma seadmest.
    
    4. Süsteemis käivitub GeMAPS tunnuste eraldamise protsess
    
    5. Süsteem töötleb failid, eraldab GeMAPS tunnused ja salvestab andmed andmebaasi.
    
    6. Kasutajale kuvatakse õnnestumise teade.
    \\ \hline
    Alternatiivne sündmuste käik & 
    - Kui kasutaja ei vali faile, kuvatakse veateade ja protsessi ei alustata.
    
    - Kui failide töötlemine ebaõnnestub, kuvatakse veateade ja protsess peatub.
    \\ \hline
    Lõpptulemused & 
    - Eduka stsenaariumi korral GeMAPS tunnused salvestatakse andmebaasi edasiseks analüüsiks ja visualiseerimiseks.
    
    - Ebaõnnestunud stsenaariumi korral töötlust ei sooritata ning kasutajale kuvatakse teavitus.
    \\ \hline
\end{longtable}

\begin{longtable}{|p{2.5cm}|p{11cm}|}
    \caption{{Salvestuste ja töödeldud andmete kustutamine.}}
    \label{tab:kasutusjuht2}\\ \hline
    \textbf{Nimi} &  \textbf{Salvestuste ja töödeldud andmete kustutamine}  \\
    \hline
    \endfirsthead
    \hline
    \textbf{Nimi} &  \textbf{Salvestuste ja töödeldud andmete kustutamine}  \\
    \hline
    \endhead
    \endfoot
    \hline
    \endlastfoot
    Kirjeldus & Kasutaja saab hallata salvestusi ja nendega seotud töödeldud andmeid, sealhulgas kustutada neid andmebaasist.\\ \hline
    Eeltingimused & Kõnesalvestused ja töödeldud andmed on olemas andmebaasis.\\ \hline
    Põhiline sündmuste käik & 
    1. Kasutaja avab salvestuste haldamise akna.
    
    2. Süsteem kuvab salvestuste nimekirja.
    
    3. Kasutaja valib salvestused, mida ta soovib kustutada.
    
    4. Süsteem kuvab dialoogi, milles küsitakse kustutamise kinnitust.
    
    5. Kasutaja kinnitab kustutamise.
    
    6. Süsteem eemaldab valitud salvestuse andmed andmebaasist ning kuvab kinnitusteate.
    \\ \hline
    Alternatiivne sündmuste käik & 
    - Kui kasutaja tühistab kustutamise kinnituse, jäävad andmed muutmata.
    \\ \hline
    Lõpptulemused & 
    - Edukas stsenaariumi korral valitud salvestuse andmed kustutatakse andmebaasist.
    
    - Ebaõnnestunud stsenaariumi korral andmed ei muutu, kuvatakse vastav teavitus.
    \\ \hline
\end{longtable}


\begin{longtable}{|p{2.5cm}|p{11cm}|}
    \caption{{Tunuste visualiseerimine.}}
    \label{tab:kasutusjuht3}\\ \hline
    \textbf{Nimi} &  \textbf{Tunnuste visualiseerimine}  \\
    \hline
    \endfirsthead
    \hline
    \textbf{Nimi} &  \textbf{Tunnuste visualiseerimine}  \\
    \hline
    \endhead
    \endfoot
    \hline
    \endlastfoot
    Kirjeldus & Kasutaja saab visualiseerida GeMAPS tunnuseid erinevatel diagrammitüüpidel, sealhulgas ajateljel, radardiagrammil, histogrammil, karpdiagrammil ja vokaalikaardil.\\ \hline
    Eeltingimused & GeMAPS tunnused on eraldatud ja salvestatud andmebaasi.\\ \hline
    Põhiline sündmuste käik & 
    1. Kasutaja avab tunnuste visualiseerimise menüü.
    
    2. Süsteem kuvab valikud tunnuste, salvestuste ja graafikutüüpide valimiseks.
    
    3. Kasutaja valib salvestuse, tunnused ja graafikutüübi.
    
    4. Kasutaja vajutab visualiseerimise nupule.
    
    5. Süsteem koostab ja kuvab valitud parameetrite alusel visualisatsiooni.
    \\ \hline
    Alternatiivne sündmuste käik & 
    - Kui valikud ei vasta nõuetele (nt pole valitud tunnuseid), kuvatakse veateade.
    \\ \hline
    Lõpptulemused & 
    - Eduka stsenaariumi korral kuvatakse valitud visualisatsioon ning kasutaja saab selle eksportida pildifailina või JSON-andmetena.
    
    - Ebaõnnestunud stsenaariumi korral visualisatsiooni ei looda ning kuvatakse teavitus.
    \\ \hline
\end{longtable}


\begin{longtable}{|p{2.5cm}|p{11cm}|}
    \caption{{Sarnaste kõnelejate leidmine.}}
    \label{tab:kasutusjuht4}\\ \hline
    \textbf{Nimi} &  \textbf{Sarnaste kõnelejate leidmine}  \\
    \hline
    \endfirsthead
    \hline
    \textbf{Nimi} &  \textbf{Sarnaste kõnelejate leidmine}  \\
    \hline
    \endhead
    \hline
    \endfoot
    \hline
    \endlastfoot
    Kirjeldus & Kasutaja saab leida kõnekorpusest salvestused, mis on kõige sarnasemad valitud sihtsalvestusega\\ \hline
    Eeltingimused & GeMAPS tunnused on eraldatud ja salvestatud andmebaasi.\\ \hline
    Põhiline sündmuste käik & 
    1. Kasutaja avab sarnasuse analüüsi valiku juhtpaneelis.
    
    2. Süsteem kuvab sihtsalvestuse ja teiste salvestuste valikuvõimalused.
    
    3. Kasutaja valib sihtsalvestuse, millele sarnaseid otsitakse ja määrab sarnasuse arvutamise valikud (nt meetod, tagastavate salvestuste arv tulemuses).
    
    4. Süsteem arvutab valitud meetodil sarnasused ja kuvab tulemused graafikuna.
    \\ \hline
    Alternatiivne sündmuste käik & 
    - Kui kasutaja ei määra sihtsalvestust, kuvatakse veateade.
    \\ \hline
    Lõpptulemused & 
    - Eduka stsenaariumi korral kuvatakse sarnaste salvestuste analüüs ja visualisatsioon.
    
    - Ebaõnnestunud stsenaariumi korral analüüsi ei sooritata, kuvatakse teavitus.
    \\ \hline
\end{longtable}

\begin{longtable}{|p{2.5cm}|p{11cm}|}
    \caption{{Helisalvestuse esitamine ja helilaine visualiseerimine.}}
    \label{tab:kasutusjuht5}\\ \hline
    \textbf{Nimi} &  \textbf{Helisalvestuse esitamine ja helilaine visualiseerimine}  \\
    \hline
    \endfirsthead
    \hline
    \textbf{Nimi} &  \textbf{Helisalvestuse esitamine ja helilaine visualiseerimine}  \\
    \hline
    \endhead
    \hline
    \endfoot
    \hline
    \endlastfoot
    Kirjeldus & Kasutaja saab kuulata kõnesalvestusi ning kuvatakse helilaine visualisatsioon.\\ \hline
    Eeltingimused & Kasutaja on lisanud soovitud kõnesalvestuste failid rakenduse kausta nimega \textit{data}\\ \hline
    Põhiline sündmuste käik & 
    1. Kasutaja avab rakenduse.
    
    2. Süsteem kuvab saadaval olevate salvestuste nimekirja.
    
    3. Kasutaja valib salvestuse ja käivitab selle esitamise.
    
    4. Süsteem esitab helisalvestuse ja kuvab helilaine visualisatsiooni.
    \\ \hline
    Alternatiivne sündmuste käik & 
    - Kui kasutaja valitud salvestus ei ole mingil põhjusel saadaval, kuvatakse veateade.
    \\ \hline
    Lõpptulemused & 
    - Eduka stsenaariumi korral salvestus esitatakse ja helilaine kuvatakse visuaalselt.
    
    - Ebaõnnestunud stsenaariumi korral esitamist ei toimu ning kuvatakse teavitus.
    \\ \hline
\end{longtable}

\begin{longtable}{|p{2.5cm}|p{11cm}|}
    \caption{{Andmete ja visualisatsioonide eksportimine.}}
    \label{tab:kasutusjuht6}\\ \hline
    \textbf{Nimi} &  \textbf{Andmete ja visualisatsioonide eksportimine}  \\
    \hline
    \endfirsthead
    \hline
    \textbf{Nimi} &  \textbf{Andmete ja visualisatsioonide eksportimine}  \\
    \hline
    \endhead
    \hline
    \endfoot
    \hline
    \endlastfoot
    Kirjeldus & Kasutaja saab eksportida visualiseeritud andmeid ja graafikuid JSON- või pildifailidena.\\ \hline
    Eeltingimused & Visualiseerimine või sarnasuse analüüs on edukalt läbi viidud.\\ \hline
    Põhiline sündmuste käik & 
    1. Kasutaja vajutab JSON faili ekspordi nupule juhtpaneelis või pildifaili ekspordi ikoonile visualisatsiooni vaate paremas nurgas.
    
    2. Kasutaja valib failitüübi ja salvestuskoha ning kinnitab ekspordi.
    
    4. Süsteem salvestab valitud failid kasutaja määratud asukohta.
    \\ \hline
    Alternatiivne sündmuste käik & 
    - Kui kasutaja ei määra salvestuskohta, ekspordi protsess katkestatakse ja kuvatakse teavitus.
    \\ \hline
    Lõpptulemused & 
    - Eduka stsenaariumi korral eksporditud fail salvestatakse määratud asukohta.
    
    - Ebaõnnestunud stsenaariumi korral ekspordi protsess katkestatakse ja kuvatakse teavitus.
    \\ \hline
\end{longtable}



