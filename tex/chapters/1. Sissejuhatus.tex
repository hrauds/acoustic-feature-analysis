Foneetikauuringutes kasutatakse kõnekorpusi, mis sisaldavad sadade inimeste kõnenäiteid. Korpuste analüüsil leitakse erinevaid akustilisi tunnuseid, nagu põhitoon (F0), formandid, spektrid ja hääle kvaliteeti kirjeldavad tunnused. Nende tunnuste abil saab tuvastada erinevaid kõnestiile, võõrkeele aktsente, emotsionaalseid ja tervislikke seisundeid.

Üheks laialt levinud tunnustekomplektiks foneetikauuringutes on GeMAPS (The Geneva Minimalistic Acoustic Parameter Set) \cite{eyben2016gemaps}. See komplekt sisaldab mimeid olulisi akustilisi tunnuseid, mida saab usaldusväärselt kasutada mitmesugustes uuringutes. Paraku ei ole hetkel kuigi palju tööriistu, mis võimaldaksid GeMAPS-tunnuste automaatset helisalvestisest eraldamist ja põhjalikumat visualiseerimist ühes rakenduses.

Korpuste analüüsi tööprotsessi osad on sageli järgmised tegevused: akustiliste tunnuste eraldamine kõnefailidest, akustiliste tunnuste visualiseerimine, uurija poolt etteantud akustiliste omadustega kõnenäidete leidmine. Kuigi on olemas mitmeid tööriistu, mis suudavad neid ülesandeid eraldi täita, puudub hetkel terviklik rakendus, mis võimaldaks täita neid kõiki neid kõnekorpuste analüüsiga seotud ülesandeid.

Selle bakalaureusetöö eesmärk on luua tööriist, mis võimaldab:
\begin{itemize}
    \item GeMAPS tunnuste eraldamist kõnefailidest,
    \item tunnuste visualiseerimist erinevatel diagrammitüüpidel: joondiagramm/ajatelg, vokaalikaart, radar, karpdiagramm ja histogramm,
    \item sarnaste kõnelejate otsingut kõnekorpusest.
\end{itemize}

Loodud rakendus on abiks uurijatele, kes soovivad kasutada GeMAPS-tunnuseid foneetika- või hääleanalüüsis. Rakendus koondab mitmed peamised uurimisprotsessi funktsioonid nagu: tunnuste eraldamine, andmete visualiseerimine ja sarnaste salvestiste leidmine.
