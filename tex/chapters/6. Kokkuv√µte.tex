Käesoleva bakalaureusetöö eesmärk oli luua kõneanalüüsi rakendus, mis võimaldab:
GeMAPS akustiliste tunnuste eraldamist helisalvestustest,
tunnuste visualiseerimist mitmesugustel diagrammitüüpidel (ajatelje-, histogrammi-, karp-, radari- ja vokaalikaardi diagrammid) ning
sarnaste salvestiste leidmist.

Helifailidest eraldatati GeMAPS madala taseme deskriptorid (LLD), kasutades OpenSMILE’i ning töödeldi kõnesalvestuste TextGrid faile märgendusinfo saamise jaoks. Tunnuste väärtused ja salvestuste andmed salvestati dokumentidena MongoDB andmebaasis.
Rakenduse kasutajaliides loodi PyQt raamistikuga. Interaktiivsete diagrammide kuvamiseks (nt ajatelg, histogramm, karpdiagramm, radar ja vokaalikaart) kasutatati Plotly teeki.
Sarnaste salvestiste leidmiseks rakendatakse klasterdamist (KMeans) ja koosinussarnasuse arvutamist PCA-ga vähendatud dimensiooniruumis kui ka mitte vähendatud dimensioonidega tunnuste arvutamist. Sarnasusanalüüsi tulemused visualiseeritakse nii tulp- kui hajuvdiagrammidel.

Loodud rakendus vastas algselt püstitatud funktsionaalsetele nõuetele. Tagasiside ja testimise põhjal hinnati rakendust kasulikuks tunnuste kiireks visualiseerimiseks ning toodi esile võimalused kasutajaliidese mugavamaks muutmiseks ning funktsionaalsuse laiendamiseks.
