\section{Funktsionaalsed nõuded}
Rakenduse arenduse alguses lepiti juhendaja kokku funktsionaalsed nõuded, et määratleda rakenduse fookus ja võimalused. 

Funktsionaalsed nõuded:
\begin{itemize}
    \item GeMAPS tunnuste eraldamine kõnefailidest. Rakendus peab suutma kõnefailidest eraldada GeMAPS akustilisi tunnuseid. Selleks tuleb töödelda salvestused ja nendega seotud TextGrid failid.
   
    \item Tunnuste visualiseerimine erinevatel diagrammitüüpidel, nagu ajatelje joondiagramm, histogramm, radar diagramm, karpdiagramm, vokaalikaart.
    
    \item Sarnaste kõnelejate leidmine kõnekorpusest.

    \item Võimalus eksportida analüüsi tulemusi: pildifaile ja andmeid.
\end{itemize}

\section{Funktsionaalsete nõuete kasutusjuhud}
Arenduse käigus koostati täpsemad nõuded, mida täpsustati pidevalt testimisel tekkinud mõtetest. Järgnevalt esitatakse kokkulepitud nõuded täpsemalt kasutusjuhtudena.

Kasutusjuhud on kirjeldatud tabelites 1-7.

\begin{longtable}{|p{2.5cm}|p{11cm}|}
    \caption{Kasutusjuht GeMAPS tunnuste eraldamine kõnefailidest}
    \label{tab:kasutusjuht1}\\ \hline
    \textbf{Nimi} &  \textbf{GeMAPS tunnuste eraldamine kõnefailidest}  \\
    \hline
    \endfirsthead
    \hline
    \textbf{Nimi} &  \textbf{GeMAPS tunnuste eraldamine kõnefailidest}  \\
    \hline
    \endhead
    \hline
    \endfoot
    \hline
    \endlastfoot
    Kirjeldus & Kasutaja saab importida kõnesalvestised ja nendega seotud TextGrid failid, et eraldada ja salvestada GeMAPS akustilisi tunnuseid.\\ \hline
    Eeltingimused & Kasutajal on .wav ja TextGrid-failid.\\ \hline
    Põhiline sündmuste käik & 
    1. Kasutaja avab kõnefailide importimise akna.
    
    2. Süsteem kuvab kõnefailide ja TextGridide üleslaadimise vormi.
    
    3. Kasutaja valib kõnefailid ja vastavad TextGridid oma seadmest.
    
    4. Kasutaja käivitab GeMAPS tunnuste eraldamise protsessi.
    
    5. Süsteem töötleb failid, eraldab GeMAPS tunnused ja salvestab andmed andmebaasi.
    
    6. Kasutajale kuvatakse õnnestumise teade.
    \\ \hline
    Alternatiivne sündmuste käik & 
    - Kui kasutaja ei vali faile, kuvatakse veateade ja protsessi ei alustata.
    
    - Kui failide töötlemine ebaõnnestub, kuvatakse veateade ja protsess peatub.
    \\ \hline
    Lõpptulemused & 
    - Edukas stsenaarium: GeMAPS tunnused salvestatakse andmebaasi edasiseks analüüsiks ja visualiseerimiseks.
    
    - Ebaõnnestunud stsenaarium: töötlust ei sooritata, kasutajale kuvatakse teavitus.
    \\ \hline
\end{longtable}

\begin{longtable}{|p{2.5cm}|p{11cm}|}
    \caption{{Salvestuste ja töödeldud andmete kustutamine}}
    \label{tab:kasutusjuht2}\\ \hline
    \textbf{Nimi} &  \textbf{Salvestuste ja töödeldud andmete kustutamine}  \\
    \hline
    \endfirsthead
    \hline
    \textbf{Nimi} &  \textbf{Salvestuste ja töödeldud andmete kustutamine}  \\
    \hline
    \endhead
    \endfoot
    \hline
    \endlastfoot
    Kirjeldus & Kasutaja saab hallata salvestusi ja nendega seotud töödeldud andmeid, sealhulgas kustutada neid andmebaasist.\\ \hline
    Eeltingimused & Kõnesalvestused ja töödeldud andmed on olemas andmebaasis.\\ \hline
    Põhiline sündmuste käik & 
    1. Kasutaja avab salvestuste haldamise akna.
    
    2. Süsteem kuvab salvestuste nimekirja.
    
    3. Kasutaja valib salvestuse(d), mida ta soovib kustutada.
    
    4. Süsteem kuvab dialoogi, milles küsitakse kustutamise kinnitust.
    
    5. Kasutaja kinnitab kustutamise.
    
    6. Süsteem eemaldab valitud salvestuse andmed andmebaasist ning kuvab kinnitusteate.
    \\ \hline
    Alternatiivne sündmuste käik & 
    - Kui kasutaja tühistab kustutamise kinnituse, jäävad andmed muutmata.
    \\ \hline
    Lõpptulemused & 
    - Edukas stsenaarium: valitud salvestuse andmed kustutatakse andmebaasist.
    
    - Ebaõnnestunud stsenaarium: andmed ei muutu, kuvatakse vastav teavitus.
    \\ \hline
\end{longtable}


\begin{longtable}{|p{2.5cm}|p{11cm}|}
    \caption{{Tunnuste visualiseerimine}}
    \label{tab:kasutusjuht3}\\ \hline
    \textbf{Nimi} &  \textbf{Tunnuste visualiseerimine}  \\
    \hline
    \endfirsthead
    \hline
    \textbf{Nimi} &  \textbf{Tunnuste visualiseerimine}  \\
    \hline
    \endhead
    \endfoot
    \hline
    \endlastfoot
    Kirjeldus & Kasutaja saab visualiseerida GeMAPS tunnuseid erinevatel diagrammitüüpidel, sealhulgas ajateljel, radar diagrammil, histogrammil, karpdiagrammil ja vokaalikaardil.\\ \hline
    Eeltingimused & GeMAPS tunnused on eraldatud ja salvestatud andmebaasi.\\ \hline
    Põhiline sündmuste käik & 
    1. Kasutaja avab tunnuste visualiseerimise menüü.
    
    2. Süsteem kuvab valikud tunnuste, salvestuste ja graafikutüüpide valimiseks.
    
    3. Kasutaja valib salvestuse, tunnused ja graafikutüübi.
    
    4. Kasutaja käivitab visualiseerimise.
    
    5. Süsteem koostab ja kuvab valitud parameetrite alusel visualiseeringu.
    \\ \hline
    Alternatiivne sündmuste käik & 
    - Kui valikud ei vasta nõuetele (nt pole valitud tunnuseid), kuvatakse veateade.
    \\ \hline
    Lõpptulemused & 
    - Edukas stsenaarium: Kuvatakse valitud visualiseering. Kasutaja saab selle eksportida pildifailina või JSON-andmetena.
    
    - Ebaõnnestunud stsenaarium: Visualiseeringut ei looda, kuvatakse teavitus.
    \\ \hline
\end{longtable}


\begin{longtable}{|p{2.5cm}|p{11cm}|}
    \caption{{Sarnaste kõnelejate leidmine}}
    \label{tab:kasutusjuht4}\\ \hline
    \textbf{Nimi} &  \textbf{Sarnaste kõnelejate leidmine}  \\
    \hline
    \endfirsthead
    \hline
    \textbf{Nimi} &  \textbf{Sarnaste kõnelejate leidmine}  \\
    \hline
    \endhead
    \hline
    \endfoot
    \hline
    \endlastfoot
    Kirjeldus & Kasutaja saab leida kõnekorpusest salvestused, mis on kõige sarnasemad valitud sihtsalvestusega\\ \hline
    Eeltingimused & GeMAPS tunnused on eraldatud ja salvestatud andmebaasi.\\ \hline
    Põhiline sündmuste käik & 
    1. Kasutaja avab sarnasuse analüüsi menüü.
    
    2. Süsteem kuvab sihtsalvestuse ja muude salvestuste valikuvõimalused.
    
    3. Kasutaja valib sihtsalvestuse ja määrab sarnasuse arvutamise valikud (nt meetod, tagastavate salvestuste arv tulemuses).
    
    4. Süsteem arvutab valitud meetodil sarnasused ja kuvab tulemused graafikuna.
    \\ \hline
    Alternatiivne sündmuste käik & 
    - Kui kasutaja ei määra sihtsalvestust, kuvatakse veateade.
    \\ \hline
    Lõpptulemused & 
    - Edukas stsenaarium: Kuvatakse sarnaste salvestuste analüüs ja visualiseerimine.
    
    - Ebaõnnestunud stsenaarium: Analüüsi ei sooritata, kuvatakse teavitus.
    \\ \hline
\end{longtable}

\begin{longtable}{|p{2.5cm}|p{11cm}|}
    \caption{{Helisalvestuse esitamine ja helilaine visualiseerimine}}
    \label{tab:kasutusjuht5}\\ \hline
    \textbf{Nimi} &  \textbf{Helisalvestuse esitamine ja helilaine visualiseerimine}  \\
    \hline
    \endfirsthead
    \hline
    \textbf{Nimi} &  \textbf{Helisalvestuse esitamine ja helilaine visualiseerimine}  \\
    \hline
    \endhead
    \hline
    \endfoot
    \hline
    \endlastfoot
    Kirjeldus & Kasutaja saab kuulata üles laaditud kõnesalvestust ning kuvatakse selle helilaine visualisatsioon.\\ \hline
    Eeltingimused & Kasutaja on laadinud soovitud kõnesalvestuste failid rakenduse kausta nimega "data"\\ \hline
    Põhiline sündmuste käik & 
    1. Kasutaja avab rakenduse.
    
    2. Süsteem kuvab saadaval olevate salvestuste nimekirja.
    
    3. Kasutaja valib salvestuse ja käivitab selle esitamise.
    
    4. Süsteem mängib helisalvestust ja kuvab samaaegselt helilaine visualiseerimise.
    \\ \hline
    Alternatiivne sündmuste käik & 
    - Kui kasutaja valitud salvestus ei ole mingil põhjusel saadaval, kuvatakse veateade.
    \\ \hline
    Lõpptulemused & 
    - Edukas stsenaarium: Salvestus mängitakse ja helilaine kuvatakse visuaalselt.
    
    - Ebaõnnestunud stsenaarium: Esitamist ei toimu, kuvatakse teavitus.
    \\ \hline
\end{longtable}

\begin{longtable}{|p{2.5cm}|p{11cm}|}
    \caption{{Andmete ja visualisatsioonide eksportimine}}
    \label{tab:kasutusjuht6}\\ \hline
    \textbf{Nimi} &  \textbf{Andmete ja visualisatsioonide eksportimine}  \\
    \hline
    \endfirsthead
    \hline
    \textbf{Nimi} &  \textbf{Andmete ja visualisatsioonide eksportimine}  \\
    \hline
    \endhead
    \hline
    \endfoot
    \hline
    \endlastfoot
    Kirjeldus & Kasutaja saab eksportida visualiseeritud andmeid ja graafikuid JSON- või pildifailidena.\\ \hline
    Eeltingimused & Visualiseerimine või sarnasuse analüüs on edukalt läbi viidud.\\ \hline
    Põhiline sündmuste käik & 
    1. Kasutaja vajutab ekspordi nupule.
    
    2. Süsteem kuvab eksporditüüpide (JSON, pilt) valiku.
    
    3. Kasutaja valib failitüübi ja salvestuskoha ning kinnitab ekspordi.
    
    4. Süsteem salvestab valitud failid kasutaja määratud asukohta.
    \\ \hline
    Alternatiivne sündmuste käik & 
    - Kui kasutaja ei määra salvestuskohta, ekspordi protsess katkestatakse ja kuvatakse teavitus.
    \\ \hline
    Lõpptulemused & 
    - Edukas stsenaarium: Eksporditud fail salvestatakse määratud asukohta.
    
    - Ebaõnnestunud stsenaarium: Ekspordi protsess katkeb ja kuvatakse teavitus.
    \\ \hline
\end{longtable}



