Selles peatükis antakse ülevaade rakenduse arenduse protsessist, mille etapid olid: akustiliste tunnuste eraldamine ja TextGrid-failide töötlemine, andmete salvestamine, kasutajaliidese arendus, visualisatsioonide loomine, kõnesalvestuste sarnasuse analüüs ning rakenduse testimine.

\section{Tunnuste eraldamine ja TextGridide töötlus}
Järgnevalt kirjeldatakse akustiliste tunnuste helifailidest eraldamise ning TextGridide töötlemise protsessi ja tehnoloogiaid.

\subsection{OpenSMILE}
Tunnuste eraldamiseks kasutati OpenSMILE \cite{opensmile} raamistikku, mille töötluse funktsioon võtab sisendiks helifaili ning annab väljundiks pandas DataFrame'i, kus iga rida akustilise tunnuse väärtus 10 millisekundilise ajaraami kohta.

\begin{longtable}{|r|r|r|r|}
    \caption{Näide helifailist eraldatud tunnuste DataFrame'ist.} \label{table:example} \\
    \hline
    \textbf{Start (s)} & \textbf{Loudness\_sma3} & \textbf{alphaRatio\_sma3} & \textbf{slope0-500\_sma3} \\
    \hline
    \endfirsthead
    \hline
    \textbf{Start (s)} & \textbf{Loudness\_sma3} & \textbf{alphaRatio\_sma3} & \textbf{slope0-500\_sma3} \\
    \hline
    \endhead
    \hline
    \endfoot
    \hline
    \endlastfoot
    0.00 & 0.120393 & -19.566683 & -0.073501 \\
    0.01 & 0.112910 & -16.829172 & -0.061550 \\
    0.02 & 0.103573 & -14.812015 & -0.051186 \\
    0.03 & 0.104770 & -16.948393 & -0.075717 \\
    0.04 & 0.105062 & -19.221598 & -0.086498 \\
    {\vdots} & {\vdots} & {\vdots} & {\vdots} \\
    9.68 & 0.107817 & -21.088533 & -0.076632 \\
    9.69 & 0.110813 & -16.712238 & -0.043914 \\
    9.70 & 0.113438 & -17.034071 & -0.041286 \\
    9.71 & 0.112685 & -16.894552 & -0.032873 \\
    9.72 & 0.113219 & -18.397005 & -0.041675 \\
\end{longtable}

OpenSMILE töötlemise funktsiooni parameetriks määratakse eraldatav tunnuste komplekt. Neid on mitmeid erinevaid, selles rakenduses kasutatakse kõige väiksemat eGeMAPS v02 akustiliste tunnuste komplekti, kus on 88 tunnust, mis jagunevad:

\begin{enumerate}
    \item Madala taseme deskriptorid (\textit{Low Level Descriptors}), mis on otse helisignaalist tuletatud tunnused.
    \item Funktsionaalsed tunnused (\textit{functionals}): statistilised näitajad, mis arvutatakse madala taseme tunnustest, näiteks keskmised väärtused ja standardhälve.
\end{enumerate}

Nendest omakorda valiti ainult madala taseme deskriptorid (kokku 25 tunnust), et tunnuste hulk oleks kasutajale hoomatav, sest visualiseerimisel keskendutakse üksikute tunnuste visualiseerimisele. Eraldatud tunnuste loetelu on esitatud Lisas 2.

\subsection{TextGridide töötlemine}
TextGrid \cite{textgrid} on failiformaat, mida kasutatakse kõnesalvestuse märgendamiseks. Lastekõne korpuse puhul, mida rakenduse arenduses näidisandmetena kasutati, on iga salvestuse TextGridis märgendatud IntervalTiers kiht, mis võib jaguneda veel näiteks: HMM-words, HMM-phonemes, cv (konsonant või vokaal) ja foot. Kihid, mida rakenduses TextGridide töötlusel kasutati on HMM-words ja HMM-phonemes. Nendes kihtides on andmed selle kohta, millal sõna või foneem algab ja lõppeb. Need andmed tuli salvestada, et võimaldada kindlate sõnade ja foneemide visualiseerimine.

Sõnade ja foneemide info töötlemiseks kasutatati praatio teeki \cite{praatio}. Sellega oli võimalik lihtsasti eraldada TextGridist kihtide infot. Praatio on Pythoni teek, mis lihtsustab TextGridide lugemist. 

\section{Andmete salvestamine MongoDB andmebaasis}
Rakenduse andmebaasi valikul kaaluti võimalustest relatsioonilisi andmebaase või dokumendipõhiseid lahendusi nagu MongoDB. MongoDB \cite{mongodb} on NoSQL andmebaas, kus andmed salvestatakse BSON-formaadis dokumentidena. MongoDB on paindlik, sest see võimaldab paindlikku andmebaasi skeemi ning ühes kollektsioonis hoida erinevate väljade ja andmetüüpidega dokumente \cite{mongodb_data_modeling}. Kuna rakenduse arenduse alguses ei olnud veel selge, kuidas ja milliseid andmeid on vaja salvestada, tundus see valikul oluline. Samuti oli MongoDB seadistamine lihtne ja kiire.
 
Andmete salvestamise protsess:
Peale tunnuste eraldamist ja  TextGridide töötlemist salvestatakse andmed kolme kollektsiooni: \textit{Recordings}, \textit{Words} ja \textit{Phonemes}.

\begin{longtable}{|p{2.5cm}|p{3cm}|p{8cm}|}
    \caption{Salvestuste kollektsiooni väljad, nende andmetüübid ja semantika.}
    \label{tab:recordings}\\
    \hline
    \textbf{Välja nimi} & \textbf{Andmetüüp} & \textbf{Semantika}\\
    \hline
    \endfirsthead
    \hline
    \textbf{Välja nimi} & \textbf{Andmetüüp} & \textbf{Semantika}\\
    \hline
    \endhead
    \endfoot
    \hline
    \endlastfoot

    \_id &
    ObjectId &
    Unikaalne identifikaator, mille MongoDB genereerib automaatselt.\\ \hline
  
    recording\_id &
    string &
    Salvestuse nimi (nt 148\_10001148\_26).\\ \hline
          
    start &
    double &
    Salvestuse algusaeg sekundites.\\ \hline
    
    end &
    double &
    Salvestuse lõppaeg sekundites.\\ \hline
    
    duration &
    double &
    Salvestise kogukestus sekundites.\\ \hline

    features &
    Object &
    Objekt, mis hoiab salvestise akustilisi tunnuseid. Sellel on kaks alamobjekti:
    \begin{itemize}
      \item mean – akustiliste tunnuste keskmised väärtused kujul tunnuse nimi: tunnuse keskmine väärtus (double).
      \item frame\_values – massiiv 10\,ms ajaraamide tunnuste väärtustest (double).
    \end{itemize}\\ \hline
\end{longtable}

Ka sõnade ja foneemide puhul salvestatakse iga sõna või foneemi dokument vastavasse kollektsiooni, kuid tunnuste puhul salvestatakse ainult keskmised väärtused sõna või foneemi ajavahemiku kohta. Enne salvestamist filtreeritakse välja mitte-kõne segmendid, nagu vaikused ja müra, mis on vastavate siltidega (nt .noise või sil) TextGridides märgistatud. Nende segmentide tunnuste väärtuseid ei ole vaja andmebaasi salvestada, sest neid ei ole vaja eraldi visualiseerida, kuna vajalik on segmentidest ainult kindlaid sõnu või foneeme visualiseerida.

\begin{longtable}{|p{2.5cm}|p{3cm}|p{8cm}|}
    \caption{Sõnade kollektsiooni väljad, nende andmetüübid ja semantika.}
    \label{tab:words}\\
    \hline
    \textbf{Välja nimi} & \textbf{Andmetüüp} & \textbf{Semantika}\\
    \hline
    \endfirsthead
    \hline
    \textbf{Välja nimi} & \textbf{Andmetüüp} & \textbf{Semantika}\\
    \hline
    \endhead
    \endfoot
    \hline
    \endlastfoot

    \_id &
    ObjectId &
    Unikaalne identifikaator, mille MongoDB genereerib automaatselt.\\ \hline

    duration &
    double &
    Sõna kestus sekundites.\\ \hline

    start &
    double &
    Sõna algamise ajahetk sekundites.\\ \hline

    end &
    double &
    Sõna lõppemise ajahetk sekundites.\\ \hline

    text &
    string &
    Sõna esitus tekstina (nt linna+tulede).\\ \hline

    recording\_id &
    string &
    Salvestuse nimi (nt 148\_10001148\_26).\\ \hline

    parent\_id &
    ObjectId &
    Salvestuse dokumendi ID, millesse sõna kuulub.\\ \hline
    
    features &
    Object &
    Objekt, mis hoiab sõna akustilisi tunnuseid:
    \begin{itemize}
      \item mean – alamdokumendis on erinevate tunnuste keskmised väärtused kujul tunnuse nimi: double.
    \end{itemize}\\ \hline

\end{longtable}


\begin{longtable}{|p{2.5cm}|p{3cm}|p{8cm}|}
    \caption{Foneemide kollektsiooni väljad, nende andmetüübid ja semantika.}
    \label{tab:phonemes}\\
    \hline
    \textbf{Välja nimi} & \textbf{Andmetüüp} & \textbf{Semantika}\\
    \hline
    \endfirsthead
    \hline
    \textbf{Välja nimi} & \textbf{Andmetüüp} & \textbf{Semantika}\\
    \hline
    \endhead
    \endfoot
    \hline
    \endlastfoot

    \_id &
    ObjectId &
    Unikaalne identifikaator, mille MongoDB genereerib automaatselt.\\ \hline

    duration &
    double &
    Foneemi kestus sekundites.\\ \hline

    start &
    double &
    Foneemi alguse ajahetk sekundites.\\ \hline

    end &
    double &
    Foneemi lõppemise ajahetk sekundites.\\ \hline

    text &
    string &
    Foneemi esitus tekstina (nt "t").\\ \hline

    parent\_id &
    ObjectId &
    Viide selle sõna dokumendi ID-le, millesse foneem kuulub.\\ \hline
    
    word\_text &
    string &
    Sõna tekst, millesse foneem kuulub (nt linna+tulede).\\ \hline

    recording\_id &
    string &
    Salvestuse nimi (nt 148\_10001148\_26).\\ \hline

    features &
    Object &
    Objekt, mis hoiab foneemi akustilisi tunnuseid:
    \begin{itemize}
      \item mean – alamdokumendis on erinevate tunnuste keskmised väärtused kujul tunnuse nimi: double.
    \end{itemize}\\ \hline
\end{longtable}

\section{Kasutajaliidese loomine PyQt raamistikuga}
PyQt \cite{pyqt5} on Pythoni graafiliste kasutajaliideste arendamise raamistik. Selles on erinevad valmiskomponentid, näiteks nupud, aknad, menüüd ja otsingukastid. PyQt raamistik toetab Windows, macOs ja Linux operatsioonisüsteeme.

Peale PyQt kaaluti veel teisi GUI loomise raamistikke nagu Tkinter \cite{tkinter}, mis on samuti Pythoni raamistik. Kuid selle funktsionaalsus on piiratud võrreldes PyQt-ga ning valmiskomponentide valik väiksem \cite{pyqt_tkinter}.

\section{Visualisatsioonide loomine}

Tunnuste visualiseerimise funktsionaalsuse eesmärk oli võimaldada kasutajal kuvada akustilisi tunnuseid mitmel erineval moel interaktiivsetel graafikutel, et andmeid oleks mugav uurida.

Rakenduse varasemates versioonides loodi graafikute visuaalid Matplotlib \cite{matplotlib} teegiga, mis on tuntud ja usaldusväärne teek andmete visualiseerimiseks, kuid kuna see pakub vaikimisi vaid staatilisi jooniseid, siis hakati otsima teisi võimalusi, mis pakuksid rohkem valmis võimalusi graafikute interaktiivsuse ja disaini poolest. 

 Katsetati ka PyQt enda graafiku loomise komponente, kuid need olid väga sarnased Matplotlibi graafikutele, interaktiivsust seadistada oli keerulisem ning need ei pakkunud piisavalt sisseehitatud funktsionaalsust.

Lõpuks valiti graafikute loomise tehnoloogiaks Plotly \cite{plotly}, sest sellega on graafikud vaikimisi interaktiivsed, näiteks on võimalik suumimine, kerimine, andmepunktide (\textit{tooltip}) lisainformatsiooni kuvamine ja pildifailide eksport.

Igale graafikutüübile tehti eraldi meetod, mis loob pandas DataFrame'ist Plotly graafikuobjekti. Graafikuobjekt teisendatakse HTML koodiks, mille PyQt QWebEngineView komponent rakenduse põhivaate aknas renderdab. Kõigile graafikutele konfigureeriti ühtne legendi, telgede ja värvide stiil. Visualiseerimise graafikutüüpidest valiti realiseerimiseks:

\begin{itemize}
    \item \textbf{Ajatelje diagramm}, mis kuvab valitud tunnuste väärtused ajas. Võimalik on nii terve salvestuse kui ka eraldi sõna ja foneemi visualiseerimine, näiteks saab visualiseerida, kuidas F0 põhisagedus läbi kahe erineva valitud sõna muutub.
    
    \item \textbf{Histogramm} võimaldab visualiseerida, kuidas üksiku tunnuse väärtused valitud salvestuses, sõnas või foneemis väärtuste vahemikes jaotuvad. Väärtuste vahemike tulpade arv leitakse Sturgesi valemiga.

    \item \textbf{Karpdiagramm} võimaldab visualiseerida valitud tunnuste põhilisi statistilisi näitajaid nagu: miinimum, maksimum, mediaan ja kvartiilid.

    \item \textbf{Radardiagramm} võimaldab mitme salvestuse, sõna või foneemi valitud tunnuste võrdlemise. Tunnused normaliseeritakse enne kuvamist min-max meetodiga (tunnuste väärtused viiakse vahemikku 0-1).

    \item \textbf{Vokaalikaart} kujutab vokaalide paiknemist F1 ja F2 sagedusruumis. Rakendatakse Lobanovi normaliseerimist, mis teisendab F1 ja F2 väärtused z-skoorideks, et erinevate kõnelejate vokaalid oleksid paremini võrreldavad \cite{norm_vowel_normalization}. Graafikul kuvatakse iga vokaalipunkt koos häälikumärgiga.

    \item \textbf{Salvestuste klastrid ja koosinuskaugus hajuvusdiagrammil} on sarnasuse analüüsi esimene visualiseerimismeetod, mis kuvab valitud salvestused eri värvi klastritena PC1 ja PC2 tasandil hajuvusdiagrammil. Eraldi sümbolitega on klastrites märgitud salvestus, millele sarnaseid otsiti ning sellele leitud kõige sarnasemad salvestused.

    \item \textbf{Salvestuste koosinussarnasused tulpdiagrammil} visualiseerib valitud salvestuste arvutatud koosinussarnasuse väärtused tulpadena.

    \item \textbf{Andmete visualiseerimine tabelites}: lisaks graafiliste kujutistele on rakenduses võimalus kuvada iga visualisatsiooni DataFrame tabelina, mis kuvatakse visualisatsiooni all.
\end{itemize}

\section{Sarnasuse analüüs}
Sarnasuse analüüs võimaldab kasutajal tunnuste põhjal leida, millised salvestused on kõige sarnasemad, ning esitada arvutatud tulemusi loeteluna tulpades või klastrite visualisatsioonina hajuvusdiagrammil. Järgnevalt kirjeldatakse meetodeid, mida rakenduse arenduses kasutati sarnasuse hindamiseks: klasterdamine ning koosinussarnasuse arvutamine. Enne sarnasuse arvutamist koostatati pandas DataFrame kõigi tunnuste salvestuste keskmiste väärtustega. See oli sisendiks kõigi rakendavate meetodite puhul.

\subsection{Klasterdamine}
Sisendiks on kõikide salvestiste eraldatud tunnuste keskmised väärtused. Iga tunnuse veerud skaleeritakse Scikit-learn StandardScaleri \cite{scikit_standardscaler} abil, mille tulemusena on nende keskmine väärtus 0 ja standardhälve 1.
Skaleeritud tunnuste hulk teisendatakse PCA meetodil vähendatud dimensioonidega ruumi.
Saadud PCA ruumis jagatakse salvestused nelja klastrisse, kasutades Scikit-learn KMeans algoritmi. Nelja klastrit kasutatakse siinses töös näitejuhuna, tegelikkuses peaks kasutama optimaalse klastrite arvu leidmiseks kindlaid meetodeid.

\subsection{Koosinussarnasuse arvutamine}
Tunnused skaleeritakse StandardScaleriga samal viisil nagu klasterdamise puhul. Peale skaleerimist arvutatakse koosinuskaugus kõigile valitud salvestustele sihtsalvestuse suhtes. Koosinuskaugus valiti, sest see sobib kõnefailide analüüsiks. Plaan oli realiseerida ka teiste kauguste arvutamise kasutamine, nagu Manhattani, Eukleidese kaugus, kuid need jäid tegemata. Lõpuks valitakse kasutaja poolt määratud arv kõige sarnasemaid salvestusi ja visualiseeritakse tulpadena

\subsection{Koosinussarnasuse arvutamine PCA-ga}
Koosinussarnasuse arvutamist koos põhikomponentide analüüsiga kasutatatakse kahe visualisatsiooni koostamisel:
hajuvusdiagrammil sarnaste punktide märkimisel
tulpdiagrammil sarnasuse väärtuste visualiseerimisel.

Peale tunnuste skaleerimist StandardScaleriga arvutatakse iga salvestise vahel koosinussarnasus PCA ruumis - sarnasust hinnatakse PCA-komponentide telgedel. PCA jätab kõrvale tunnused, mille varieeruvus on väike. Seega, kui on palju tunnuseid ja osa neist ei kanna kasulikku infot, siis need eemaldatakse.

Lõpuks valitakse kasutaja määratud arv kõige väiksema koosinuskaugusega ehk kõige sarnasemad salvestused ja kuvatakse nende koosinussarnasus tulpdiagrammil või hajuvusdiagrammil eraldi sümbolitega.

\section{Testimine}
Rakendust testiti mitmetel kordadel koos juhendajaga. Rakenduse arenduse lõppedes tehti üks põhjalikum testimine kasutades rakenduse funktsionaalsuse kasutusjuhte.
Testimise eesmärgid olid:
\begin{itemize}
    \item Kasutatavuse kontrollimine
    \item Funktsionaalsuse kohta tagasiside saamine
    \item Tehniliste probleemide avastamine
\end{itemize}

\subsection{Testitav rakendus}
Tagasiside küsimiseks ning kasutajatele testimiseks loodi rakendusest PyInstalleriga \cite{pyinstaller_manual} pakendatud ja käivitatav versioon. Kuna andmebaasi ei seadistatud serverisse, siis oleksid kasutajad pidanud rakenduse kasutamiseks oma arvutis MongoDB andmebaasi seadistama. Seadistamis vaeva vähendamiseks kasutab pakendatud versioon MongoMock \cite{mongomock} tööriista, mis on mälu-põhine andmebaasiteek ja ei vaja eraldi paigaldust. MongoMock simuleerib MongoDB päringuid ja töötab näidisandmetega, mis testrakendusele lisati. See võimaldab kasutada kõiki rakenduse funktsioone samamoodi nagu päris MongoDB andmebaasiga. Ainuke erinevus on, et kui uusi andmeid importida, siis need on ajutised ja kaovad kui programm lõpetatakse. Testrakendusele lisati kaasa JSON formaadis näidisandmed ja .wav failid koos näidis helisalvestustega.
